\hypertarget{index_intro_sec}{}\section{Introduction}\label{index_intro_sec}
N\+E\+OC is a C and C++ library that allows low level and extremely fast control over the Udoo Neo. N\+E\+OC allows control over all internal and brick aspects of the Udoo. This library is easy to install Since it\textquotesingle{}s already precompiled and requires only one header for all the functions. Note, the C++ classes are Not supported in C and you will have to use the longer naming convention. ~\newline
~\newline
So far this is what is supported\+: 
\begin{DoxyEnumerate}
\item {\bfseries G\+P\+IO} on all available bank G\+P\+IO pins 
\item {\bfseries P\+WM} on all enabled pwm pins (Enable via device tree editor) 
\item {\bfseries F\+A\+KE P\+WM} on all available bank G\+P\+IO pins 
\item {\bfseries A\+C\+C\+EL} raw data and calibrated data with custom poll rates 
\item {\bfseries M\+A\+G\+NO} same as accel just a magnetometer (Magno for short) 
\item {\bfseries G\+Y\+RO} same as accel just gyroscope 
\item {\bfseries T\+E\+MP} The I2C temperature brick sensor plugin 
\item {\bfseries B\+U\+I\+L\+T\+IN} Control over builtin tools like disabling the m4 core and setting graphics options (Specific to Udoo Neo) 
\item {\bfseries S\+E\+R\+VO} Currently experiemental but should work on any G\+P\+IO pin to write between 0 and 180 degrees on a servo 
\end{DoxyEnumerate}~\newline
 The reason for this library to be written in C and have C++ support (using extern), was mainly for the speed Of the compiled low level languages and they\textquotesingle{}re relatively easy to port up into higher level languages. That\textquotesingle{}s Why I need your help to manage and port this fast base into higher languages. I have too many other projects to Work on and this was initially just made for a personal project. Then realized why not just add some documentation And make it a library. To install just scroll down to the installation section.

~\newline
 \hypertarget{index_install_sec}{}\section{Installation}\label{index_install_sec}
~\newline
\subsubsection*{Easy Installation}

~\newline
 The installation for N\+E\+OC is extremely easy. For the quick installation method just copy and paste the below command~\newline
 {\ttfamily wget -\/q -\/O -\/ \href{https://raw.githubusercontent.com/smerkousdavid/NEOC.GPIO/install.sh}{\tt https\+://raw.\+githubusercontent.\+com/smerkousdavid/\+N\+E\+O\+C.\+G\+P\+I\+O/install.\+sh} $\vert$ bash}~\newline
 \subsubsection*{Full Installation}

~\newline
\hypertarget{index_sec}{}\section{License}\label{index_sec}
~\newline
This Software may be used commercially if credit is provided to original owner in private source~\newline
~\newline
 \tabulinesep=1mm
\begin{longtabu} spread 0pt [c]{*{1}{|X[-1]}|}
\hline
{\itshape  {\bfseries N\+E\+OC} (libneo) is free software\+: you can redistribute it and/or modify~\newline
 ~~~~it under the terms of the G\+NU General Public License as published by~\newline
 ~~~~the Free Software Foundation, either version 3 of the License, or~\newline
 ~~~~(at your option) any later version.~\newline
 ~\newline
 {\bfseries N\+E\+OC} (libneo) is distributed in the hope that it will be useful,~\newline
 ~~~~but W\+I\+T\+H\+O\+UT A\+NY W\+A\+R\+R\+A\+N\+TY; without even the implied warranty of~\newline
 ~~~~M\+E\+R\+C\+H\+A\+N\+T\+A\+B\+I\+L\+I\+TY or F\+I\+T\+N\+E\+SS F\+OR A P\+A\+R\+T\+I\+C\+U\+L\+AR P\+U\+R\+P\+O\+SE. See the~\newline
 ~~~~G\+NU General Public License for more details.~\newline
}

{\itshape You should have received a copy of the {\bfseries G\+NU General Public License}~\newline
 ~~~~along with this program. If not, see \href{http://www.gnu.org/licenses}{\tt http\+://www.\+gnu.\+org/licenses} ~\newline
 }\\\cline{1-1}
\end{longtabu}
